\documentclass[10pt,-letter paper]{article}
\usepackage[left=1in, right=0.75in, top=1in, bottom=0.75in]{geometry}
\usepackage{graphicx} % Required for inserting images
\usepackage{siunitx}
\usepackage{setspace}
\usepackage{gensymb}
\usepackage{xcolor}
\usepackage{caption}
%\usepackage{subcaption}
\doublespacing
\singlespacing
\usepackage[none]{hyphenat}
\usepackage{amssymb}
\usepackage{relsize}
\usepackage[cmex10]{amsmath}
\usepackage{mathtools}
\usepackage{amsmath}
\usepackage{commath}
\usepackage{amsthm}
\interdisplaylinepenalty=2500
%\savesymbol{iint}
\usepackage{txfonts}
%\restoresymbol{TXF}{iint}
\usepackage{wasysym}
\usepackage{amsthm}
\usepackage{mathrsfs}
\usepackage{txfonts}
\let\vec\mathbf{}
\usepackage{stfloats}
\usepackage{float}
\usepackage{cite}
\usepackage{cases}
\usepackage{subfig}
%\usepackage{xtab}
\usepackage{longtable}
\usepackage{multirow}
%\usepackage{algorithm}
\usepackage{amssymb}
%\usepackage{algpseudocode}
\usepackage{enumitem}
\usepackage{mathtools}
%\usepackage{eenrc}
%\usepackage[framemethod=tikz]{mdframed}
\usepackage{listings}
%\usepackage{listings}
\usepackage[latin1]{inputenc}
%%\usepackage{color}{   
%%\usepackage{lscape}
\usepackage{textcomp}
\usepackage{titling}
\usepackage{hyperref}
%\usepackage{fulbigskip}   
\usepackage{tikz}
\usepackage{graphicx}
\lstset{
	  frame=single,
	    breaklines=true
    }
\let\vec\mathbf{}
\usepackage{enumitem}
\usepackage{graphicx}\usepackage{siunitx}
\let\vec\mathbf{}
\usepackage{enumitem}
\usepackage{graphicx}
\usepackage{enumitem}
\usepackage{tfrupee}
\usepackage{amsmath}
\usepackage{amssymb}
\usepackage{mwe} % for blindtext and example-image-a in example
\usepackage{wrapfig}
\graphicspath{{figs/}}
\providecommand{\cbrak}[1]{\ensuremath{\left\{#1\right\}}}
\providecommand{\brak}[1]{\ensuremath{\left(#1\right)}}
\newcommand{\sgn}{\mathop{\mathrm{sgn}}}
\providecommand{\abs}[1]{\left\vert#1\right\vert}
\providecommand{\res}[1]{\Res\displaylimits_{#1}} 
\providecommand{\norm}[1]{\left\lVert#1\right\rVert}
%\providecommand{\norm}[1]{\lVert#1\rVert}
\providecommand{\mtx}[1]{\mathbf{#1}}
\providecommand{\mean}[1]{E\left[ #1 \right]}
\providecommand{\fourier}{\overset{\mathcal{F}}{ \rightleftharpoons}}
%\providecommand{\hilbert}{\overset{\mathcal{H}}{ \rightleftharpoons}}
\providecommand{\system}{\overset{\mathcal{H}}{ \longleftrightarrow}}
%\newcommand{\solution}[2]{\textbf{Solution:}{#1}}
 %\newcommand{\solution}{\noindent \textbf{Solution: }}
\newcommand{\cosec}{\,\text{cosec}\,}
    \providecommand{\dec}[2]{\ensuremath{\overset{#1}{\underset{#2}{\gtrless}}}}
    \newcommand{\myvec}[1]{\ensuremath{\begin{pmatrix}#1\end{pmatrix}}}
	    \newcommand{\myaugvec}[2]{\ensuremath{\begin{amatrix}{#1}#2\end{amatrix}}}
		    \newcommand{\mydet}[1]{\ensuremath{\begin{vmatrix}#1\end{vmatrix}}}
\title{MATHEMATICS}
\begin{document}

\maketitle

\begin{enumerate}
\section{differentiation}
\item $y=(\sin^{-1}x)+(\cos^{-1}x)$,then find $\frac{dy}{dx}$
\item write the order and degree of the differential equation
                              
	$(\frac{d^4y}{dx^4})^2 = (x+(\frac{dy}{dx})^2)^3$
  \item if $y=\brak{\sec^{-1}x}^2$,$x>0$,show that $x^2{\brak{x^2-1}{\frac{d^2y}{dx^2}}+{\brak{2x^3-x}}{\frac{dy}{dx}}}-2=0$

\section{functions}	  
\item if $\ast$ defined on the set $\mathbb{R}$ of all real numbers by $\ast:a \ast b =\sqrt{a^2+b^2}$,find the identity element,if it exists in $\mathbb{R}$ with respect to $\ast$
\section{matrices}
\item if $A=\myvec{0&2\\3&-4}$
   $KA=\myvec{0&3a\\2b&24}$,
then find the values of k,a and b.
\item using the properties of determinants,prove that                                                                                        $\mydet{b+c & a & a \\                                                     b   & c+a & b \\                                                  c   & c   & a+b}= 4abc$
\item find the inverse of the following matrix,using elementary transformations:                                             
	$A=\myvec{2&3&1\\2&4&1\\3&7&2}$
\item if $A=\myvec{1&1&1\\0&1&3\\1&-2&1}$,
find $A^{-1}$ hence solve the following system of equations:                             $x+y+z=6,                                                          y+3z=11                                                      and x-2y+z=0 $
\section{algebra}
\item if $tan^{-1}x-cot^{-1}x=tan^{-1}\frac{1}{\sqrt{3}}$,$x>0$,find the value of $x$ and hence find the value of $sec^{-1}\brak{\frac{2}{x}}$
	
\section{conics}
\item find the equations of the tangent and the normal to the curve  $y=\frac{x-7}{\brak{x-2}\brak{x-3}}$ at the point where it cuts the x-axis.
	
\section{integration}
\item Find $\int\frac{\sin{2x}}{\brak{\sin^2{x+1}\brak{\sin^2 {x+3}}}}$
\item Prove that 
	\begin{align*}
		\int_{a}^{b} f\brak{x}dx= \int_{a}^{b} f\brak{a+b-x}dx
	\end{align*}
		
\item $\int_{\frac{\pi}{6}}^{\frac{\pi}{3}} \frac{1}{1+\sqrt{\tan{x}}},dx$
\item find the area of the triangle whose vertices are $\brak{-1,1    },\brak{0,5},\brak{3,2}$,using integration                  
\item find the area of the region bounded by the curves $\brak{x-1}^2+y^2=1$ and $x^2+y^2=1$,using integration

\section{geometry}
\item show that the height of a cylinder,which is open at the top,having a given surface area and greatest volume ,is equal to the radius of its base

\section{vectors}
\item find the vector and cartesian equations of the plane passing through the points $\brak{2,5,-3},\brak{-2,-3,5} and \brak{5,3,-3}$.also find the point of intersection of this plane with the line passing through points $\brak{3,1,5}$ and $\brak{-1 ,-3,-1}$.
\item find the equation of the plane passing through the intersection of the planes $\overrightarrow{r}. \brak{\hat{i}+\hat{j}+\hat{k}}=1$ and $\overrightarrow{r} .\brak{2\hat{i}+3\hat{j}-\hat{k}}+4=0$ and parallel to x-axis.Hence,find the distance of the plane from x-axis 
\item let $\vec{a}$,$\vec{b}$ and $\vec{c}$ be three vectors such that $\mydet{\overrightarrow{a}}=1$,$\mydet{\overrightarrow{b}}=2$and $\mydet{\overrightarrow{c}}=3$.if the projection of $\vec{b}    $ along $\vec{a}$ is equal to the projection of $\vec{c}$ along $\    vec{a}$; and $\vec{b}$,$\vec{c}$,are perpendicular to each other,     then find $\mydet{3\overrightarrow{a}-2\overrightarrow{b}+2\overrightarrow{c}}$                             
\item find the values of $\lambda$ for which the following lines are perpendicular to each other:$\frac{x-5}{5\lambda+2}=\frac{2-y}{5}=\frac{1-z}{-1}$ ; $\frac{x}{1}=\frac{y+\frac{1}{2}}{2\lambda}=\frac{z-1}{3}$ hence,find whether the lines intersect or not

\section{probability}
\item There are two boxes $I$ and $II$.Box $I$ contains $3$ red and $6$ black balls.Box $II$ contains $5$ red and $'n'$ black balls.One of the two boxes,box $I$ and box $II$ is selected at random and a ball is drawn at random.The ball drawn is found to be red .If the probability that this red red ball comes out from box $II$ is $\frac{3}{5}$,find the value of $'n'$.

\section{linear programming}
\item A company manufactures two types of novelty souvenirs made of plywood.Souvenirs of type $A$ require $5$ minutes each for cutting and $10$ minutes each for assembling.Souvenirs of type $B$ require $8$ minutes each for cutting and $8$ minutes each for assembling.There are $3$ hours and $20$ minutes available for cutting and $4$ hours available for assembling.The profit is\rupee $50$ each for type $A$ and \rupee $60$ each for type $B$ souvenris.How many souvenris of each type should the company manufacture in order to maximize profit?Formuate the above LPP and solve it graphically and also find the maximum profit.
\end{enumerate}
\end{document}
